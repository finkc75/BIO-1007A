\documentclass{article}

\usepackage{Sweave}
\begin{document}
\Sconcordance{concordance:test.tex:test.Rnw:%
1 2 1 1 0 36 1 1 2 1 0 2 1 3 0 1 2 5 1}



\section{First header}
\subsection{Second header}
\subsubsection{Third header}

Plain text is much the same, even in \LaTeX.

Here is \textbf{bold face}\\
Here is \textit{italic text}\\
Here is \texttt{plain text}.

``Use double back ticks and single quotes to start and finish a quotation.''

\begin{itemize}
  \item first bullet
  \item second bullet
  \item third bullet
\end{itemize}

\begin{enumerate}
  \item number one
  \item number two
  \item number three
\end{enumerate}

\begin{verbatim}
This is more convenient for a larger block of verbatim text. 
This is equivalent to the ``` plain fencing ``` from markdown.
\end{verbatim}

\begin{quote}
Again, works better for a much more extensive paragraph of quoted text such as this.
\end{quote}

\begin{Schunk}
\begin{Sinput}
> x <- runif(10)
> y <- runif(10)
> plot(x,y)
\end{Sinput}
\end{Schunk}

% In contrast to R and shell scripts, the character for a comment in LaTeX is the % sign.

Use the ``escape'' sequence to show a percentage sign like this: \%.

\end{document}
